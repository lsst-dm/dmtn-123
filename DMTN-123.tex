\documentclass[DM,authoryear,toc]{lsstdoc}
% lsstdoc documentation: https://lsst-texmf.lsst.io/lsstdoc.html

\input{meta}

% Package imports go here.

% Local commands go here.

% To add a short-form title:
% \title[Short title]{Title}
\title{Batch Production Services Design}

% Optional subtitle
% \setDocSubtitle{A subtitle}

\author{%
Michelle Gower
and
Kian-Tat Lim
}

\setDocRef{DMTN-123}
\setDocUpstreamLocation{\url{https://github.com/lsst-dm/dmtn-123}}

\date{\vcsDate}

% Optional: name of the document's curator
\setDocCurator{Michelle Gower}

\setDocAbstract{%
The LSST DM Batch Production Services are designed to allow large-scale workflows, including the Data Release and Calibration Product Productions, to execute in well-managed fashion, potentially in multiple environments.
They ensure that provenance is captured and recorded to understand what environment and parameters were used to produce any dataset.
}

% Change history defined here.
% Order: oldest first.
% Fields: VERSION, DATE, DESCRIPTION, OWNER NAME.
% See LPM-51 for version number policy.
\setDocChangeRecord{%
  \addtohist{1}{2019-07-08}{Initial version extracted from LDM-152.}{Kian-Tat Lim}
}

\begin{document}

% Create the title page.
\maketitle

\section{Introduction}\label{introduction}

This document describes the baseline design of the LSST battch production
services, including the following components:

\begin{itemize}
	\item Workload/Workflow Management
	\item Processing Control
\end{itemize}

Workload and Workflow Management interfaces with the Task Framework from the Data Management Middleware \citedsp{LDM-152} to sequence
the execution of dataflow graphs across one or more distributed computational
environments.  Processing Control uses the Workload and Workflow Management
tools to execute campaigns (applications of pipelines with specific
configurations to sets of data) in an efficient, fault-tolerant manner while
monitoring the state of execution.

The substantial
computational and bandwidth requirements of the LSST Data Management
System (DMS) force the designs to be conscious of performance,
scalability, and fault tolerance.

Use cases for the Batch Production Services are given in \citeds{LDM-633} and requirements are defined in \citeds{LDM-636}.

Figure~\ref{fig:mwandinfra} illustrates how various parts of the middleware
interact with each other.

\begin{figure}
\centering
\includegraphics[width=\textwidth]{images/MiddlewareInfrastructure.pdf}
\caption{Data Management Middleware and Infrastructure}
\label{fig:mwandinfra}
\end{figure}

\section{Workload/Workflow Management}\label{workload-workflow-management}

The Workload and Workflow Management component provides management of the
execution of science payloads ranging from a single pipeline to a series of
``campaigns'', each consisting of multiple pipelines. Its services are able to
handle massively distributed computing, executing jobs when their inputs become
available and tracking their status and outputs. They ensure that the data
needed for a job is accessible to it and that outputs (including log files, if
any) are preserved. They can allocate work across multiple computing
environments, in particular between NCSA and the Satellite Computing Facility
at CC-IN2P3.

This component invokes SuperTasks (section~\ref{supertask}) via Activators to
sequence the execution of dataflow graphs across one or more distributed
computational environments, in particular compute resources allocated via a
batch computing service.

\subsection{Batch Computing}\label{batch-computing}

Batch computing occurs on LSST dedicated computing platforms at NCSA and
CC-IN2P3 and potentially on other platforms. Resources other than local CPU and
storage for computation, such as curation storage to hold final data products
(the Data Backbone, section~\ref{data-backbone}) and network connectivity, are
also needed to completely execute a pipeline and completely realize the data
handling scheme for input and output datasets; the Workload and Workflow
component utilizes these as well.

Computing resources are physical items which are not always fit for use. They
have scheduled and unscheduled downtimes and may have scheduled availability.
The management of campaigns requires the detection of unscheduled downtimes of
resources, recovery of executing pipelines affected by unscheduled downtimes,
and arranging for the best use of available resources.

One class of potential resources are opportunistic resources which may be very
capacious but may not guarantee that jobs run to completion. These resources
may be needed in contingency circumstances. The workload management system is
capable of differentiating ``kills'' from other failures so as to enable use of
these resources.

\subsection{Workflow Management}\label{workflow-management}

The Workflow Management service provides for orchestration and execution of
pipelines.  Its basic functionality is as follows:
\begin{itemize}
	\item Pre-job context:
		\begin{itemize}
			\item Supports pre-handling of any input pipeline data sets when in-job context for input data is not required.
			\item Pre-stages input data into a platform’s storage system, if available.
			\item Produces condensed versions of database tables into portable lightweight format when required.
			\item Deals with platform-specific edge services.
			\item Handles identities and provides for local identity on the computing platforms.
			\item Provides credentials and end-point information for any needed LSST services.
		\end{itemize}
	\item In-job context:
		\begin{itemize}
			\item Provides stage-in for any in-job pipeline input data sets.
			\item Provides any Butler configurations necessarily provided from in-job context.
			\item Invokes the pipeline and collects pipeline output status and other operational data.
			\item Provides stage-out for pipeline output data sets when stage-out requires job context.
		\end{itemize}
	\item Post-job context:
		\begin{itemize}
			\item Ingests any designated data into database tables.
			\item Arranges for any post-job stage out from cluster file systems.
			\item Arranges for detailed ingest into custodial data systems.
			\item Transmits job status to workload management.
		\end{itemize}
\end{itemize}

The baseline design for the Workflow Management service uses a workflow tool
such as Pegasus \citep{Pegasus} together with HTCondor \citep{HTCondor}, custom
Activators, and Data Backbone interface scripts.

\subsection{Workload Management}\label{workload-management}

Workload Management considers the ensemble of available compute resources and
the ensemble of campaigns to be executed and dispatches pipeline invocations to
the Workload orchestration system based on resource availability and campaign
priority.  It considers pipeline failures reported by the Workload
orchestration system, distinguishing errors from computing resources and
computational errors where possible, and arranges for incident reports and
retrying failed invocations when appropriate.  It exposes the progress of each
campaign to operations staff and monitoring systems, providing appropriate
logging and events.

The prototype for the software implementing this service is in GitHub repository \texttt{lsst-dm/ctrl\_bps}.


\appendix
% Include all the relevant bib files.
% https://lsst-texmf.lsst.io/lsstdoc.html#bibliographies
\section{References} \label{sec:bib}
\bibliography{local,lsst,lsst-dm,refs_ads,refs,books}

% Make sure lsst-texmf/bin/generateAcronyms.py is in your path
\section{Acronyms} \label{sec:acronyms}
\input{acronyms.tex}

\end{document}
